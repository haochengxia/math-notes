\chapter{函数变换}\label{ch 2}
\begin{note}
改变函数形式的方法包括保持原有变量,改变表达式形式,
\begin{align*}
    2 - 3z + z^2 &\rightarrow (1-z)(2-z)\\
    \frac{1}{\sqrt{1 + z^2} - z} &\rightarrow \sqrt{1 + z^2} + z
\end{align*}
以及替换变量,也称换元,例如用$y$替换$a-z$,$$a^4 - 4a^3z + 6a^2z^2 - 4az^3 + z^4 \rightarrow y^4$$
本章仅考虑不换元的\textbf{改写表达式}替换。

往往可以将整函数分解成因式,对于关于变量$z$的函数,使用$z$的最高次数来区别因式。最高次为1的为线性因式,形如,
$$f + gz$$
而最高次为2的为二次因式,形如,
$$f + gz + hz^2$$
以此类推,n次因式是n个线性因式的积。

对于$z$的整函数$Z$,求出方程$Z=0$的所有根即为求出了所有线性因式。若$z=f$为根,即$z-f$除得尽$Z$,也即为因式。若$z = f,g,h,\cdots$为根,即$Z$的乘积形式为,
$$Z = (z-f)(z-g)(z-h)\cdots$$

线性因式可实可虚,实根给出实因式,虚根给出虚因式,根据\autoref{ch 1}若函数$Z$有虚因式,则其个数必为偶数。函数$Z$的全部虚因式之积必定为实因式,因为其等于$\frac{Z}{P}$,$P$为$Z$的全体实因式之积。

若Q是4个虚线性因式之积,则可表示为两个实二次因式之积。Q形如,
$$z^4 + Az^3 + Bz^2 + Cz + D$$

\begin{proof}
假定Q不能表示成两个实因式的乘积,那么一定可以表示成以下两个虚因式之积,
\begin{align*}
    z^2 &- 2(p + q \sqrt{-1})z + r + s\sqrt{-1}\\
    z^2 &- 2(p - q \sqrt{-1})z + r - s\sqrt{-1}
\end{align*}
由以上两个虚二次因式又可以得到以下4个虚线性因式,
\begin{align*}
    I&.z - (p + q \sqrt{-1}) + \sqrt{p^2 + 2pq\sqrt{-1} - q^2 - r - s\sqrt{-1}}\\
    II&.z - (p + q \sqrt{-1}) - \sqrt{p^2 + 2pq\sqrt{-1} - q^2 - r - s\sqrt{-1}}\\
    III&.z - (p - q \sqrt{-1}) + \sqrt{p^2 - 2pq\sqrt{-1} - q^2 - r + s\sqrt{-1}}\\
    IV&.z - (p - q \sqrt{-1}) - \sqrt{p^2 - 2pq\sqrt{-1} - q^2 - r + s\sqrt{-1}}
\end{align*}
而I、III之积,II、IV之积均为实。
\end{proof}

由于虚线性因式个数是偶数,而只有两个虚线性因式时,其乘积为实二次因式,上述证明又验证了4个虚线性因式时,其可表示为两个实二次因式,由此不妨得出猜想(此处未严格证明):$z$的整函数可以表示成实线性因式和实二次因式之积。
\end{note}
