\chapter{函数变换}\label{ch 2}
\begin{note}
改变函数形式的方法包括保持原有变量,改变表达式形式,
\begin{align*}
    2 - 3z + z^2 &\rightarrow (1-z)(2-z)\\
    \frac{1}{\sqrt{1 + z^2} - z} &\rightarrow \sqrt{1 + z^2} + z
\end{align*}
以及替换变量,也称换元,例如用$y$替换$a-z$,$$a^4 - 4a^3z + 6a^2z^2 - 4az^3 + z^4 \rightarrow y^4$$
本章仅考虑不换元的\textbf{改写表达式}替换。

往往可以将整函数分解成因式,对于关于变量$z$的函数,使用$z$的最高次数来区别因式。最高次为1的为线性因式,形如,
$$f + gz$$
而最高次为2的为二次因式,形如,
$$f + gz + hz^2$$
以此类推,n次因式是n个线性因式的积。

对于$z$的整函数$Z$,求出方程$Z=0$的所有根即为求出了所有线性因式。若$z=f$为根,即$z-f$除得尽$Z$,也即为因式。若$z = f,g,h,\cdots$为根,即$Z$的乘积形式为,
$$Z = (z-f)(z-g)(z-h)\cdots$$

线性因式可实可虚,实根给出实因式,虚根给出虚因式,根据\autoref{ch 1}若函数$Z$有虚因式,则其个数必为偶数。函数$Z$的全部虚因式之积必定为实因式,因为其等于$\frac{Z}{P}$,$P$为$Z$的全体实因式之积。

若Q是4个虚线性因式之积,则可表示为两个实二次因式之积。Q形如,
$$z^4 + Az^3 + Bz^2 + Cz + D$$

\begin{proof}
假定Q不能表示成两个实因式的乘积,那么一定可以表示成以下两个虚因式之积,
\begin{align*}
    z^2 &- 2(p + q \sqrt{-1})z + r + s\sqrt{-1}\\
    z^2 &- 2(p - q \sqrt{-1})z + r - s\sqrt{-1}
\end{align*}
由以上两个虚二次因式又可以得到以下4个虚线性因式,
\begin{align*}
    I&.z - (p + q \sqrt{-1}) + \sqrt{p^2 + 2pq\sqrt{-1} - q^2 - r - s\sqrt{-1}}\\
    II&.z - (p + q \sqrt{-1}) - \sqrt{p^2 + 2pq\sqrt{-1} - q^2 - r - s\sqrt{-1}}\\
    III&.z - (p - q \sqrt{-1}) + \sqrt{p^2 - 2pq\sqrt{-1} - q^2 - r + s\sqrt{-1}}\\
    IV&.z - (p - q \sqrt{-1}) - \sqrt{p^2 - 2pq\sqrt{-1} - q^2 - r + s\sqrt{-1}}
\end{align*}
而I、III之积,II、IV之积均为实。
\end{proof}

由于虚线性因式个数是偶数,而只有两个虚线性因式时,其乘积为实二次因式,上述证明又验证了4个虚线性因式时,其可表示为两个实二次因式,由此不妨得出猜想(此处未严格证明):$z$的整函数可以表示成实线性因式和实二次因式之积。

整函数的连续性,如果$z$的整函数$Z$在$z=a$时取值为$A$,在$z=b$时取值为$B$,那么对于一个在$A$和$B$之间的值$C$,一定存在一个$c$使得$z=c$时$Z=C$。因此若表达式$Z - A = 0$和$Z - B = 0$各有一个实线性因式,那么对$Z - C = 0$也有一个实线性因式。

如果最高次式是奇数,那么函数Z至少有一个实线性因式,可以从虚根为偶数个出发得证。另一种证明方法如下。
\begin{proof}
    $z = \infty$时,除了最高次的$z^{2n+1}$其他项可忽略,从而$Z - \infty$有因式$z - \infty$。$z = -\infty$时类似。因此若$Z=C$,且$C$在$-\infty$和$+\infty$之间时,必有实数$c$使得$Z - C = 0$。
\end{proof}
进一步可以得到最高次为奇数的情况,实线性因式的个数为奇数,因为若存在偶数个实线性因式的情况下,$Z$分离出这两个因式后最高次再次变为奇数,又至少有一个实线性因式。

而如果最高次为偶数,且常数A的符号为负,则$Z$至少有两个实线性因式,此时Z形如,
$$z^{2n} \pm \alpha z^{2n-1} \pm \beta z^{2n-2} \pm \cdots \pm \gamma z - A$$
\begin{proof}
    令$z=0$,$Z=-A$;令$Z=\pm \infty$,$Z = \infty$。又$-A < 0$,一定存在实根在$-\infty$到$0$上使得$Z=0$,在$0$到$\infty$同理。
\end{proof}

在分数函数中,若分子中$z$的最高次不小于分母的,则该函数(假分数函数)可表示为一个整函数和一个新的分数函数之和。新的分数函数中,分子最高次小于分母的。
$$\frac{1+z^4}{1+z^2} = z^2 - 1 + \frac{2}{1+z^2}$$

分母是两个互质因式(即这两个因式的因式互不相同)乘积的分数函数,可分解为分别以这两个因式作为分母的两个分数函数的和。例如,
$$\frac{1 - 2z + 3z^2 - 4z^3}{1 + 4z^4}$$
分母可分解为$1 + 2z + 2z^2$和$1 - 2z + 2z^2$。
假设有,
$$\frac{1+z^4}{1+z^2} = z^2 - 1 + \frac{2}{1+z^2} = \frac{\alpha + \beta z}{1 + 2z + 2z^2} + \frac{\gamma + \delta z}{1 - 2z + 2z^2}$$
可以得到,
\begin{gather*}
    \alpha + \gamma = 1\\
    -2\alpha + \beta + 2\gamma + \delta = -2\\
    2\alpha - 2\beta + 2\gamma + 2\delta = 3\\
    2\beta + 2\delta = -4
\end{gather*}
从而,
$$\frac{1+z^4}{1+z^2} = z^2 - 1 + \frac{2}{1+z^2} = \frac{\frac{1}{2} - \frac{5}{4} z}{1 + 2z + 2z^2} + \frac{\frac{1}{2} - \frac{4}{3} z}{1 - 2z + 2z^2}$$
由于引进的未知数个数恒等于分子的项数因此一定有解。

分数函数$\frac{M}{N}$可以被分解成$N$的不相同线性因式个数那么多个简分式,形如$\frac{A}{p - qz}$。例如,
$$\frac{1 + z^2}{z - z^3} = \frac{1}{z} + \frac{1}{1-z} - \frac{1}{1+z}$$
分母$N$的每一个线性因式都对应得到一个分数函数$\frac{M}{N}$分解式中的一个简分式,下面给出单个简分式的求法:
\begin{align*}
    \intertext{设$p-qz$是$N$的一个线性因式,即}
    N = (p - qz)S
    \intertext{此处S是z的整函数,把两个因式对应的分式记作$\frac{A}{p - qz}$和$\frac{P}{S}$,有}
    \frac{M}{N} = \frac{A}{p - qz} + \frac{P}{S} = \frac{M}{(p - qz)S}
    \intertext{即}
    \frac{P}{S} =  \frac{M - AS}{(p - qz)S}
    \intertext{可知$q - qz$是$M - AS$的因式。$z = \frac{p}{q}$时$A = \frac{M}{S}$,也就是说$A$等于$z = \frac{p}{q}$时,$\frac{M}{S}$的值。}
\end{align*}
应用实例,
$$\frac{1 + z^2}{z - z^3}$$
取线性因式$z$,那么此时$S = 1-z^2, M = 1 + z^2$,可得$A=1$。

\end{note}